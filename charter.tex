\documentclass[11pt]{charter}

% El títulos de la memoria, se usa en la carátula y se puede usar el cualquier lugar del documento con el comando \ttitle
\titulo{People Behavior Tracking (PBT)} 

% Nombre del posgrado, se usa en la carátula y se puede usar el cualquier lugar del documento con el comando \degreename
%\posgrado{Carrera de Especialización en Sistemas Embebidos} 
%\posgrado{Carrera de Especialización en Internet de las Cosas} 
\posgrado{Carrera de Especialización en Inteligencia Artificial}
%\posgrado{Maestría en Sistemas Embebidos} 
%\posgrado{Maestría en Internet de las cosas}

% Tu nombre, se puede usar el cualquier lugar del documento con el comando \authorname
\autor{Hernán Contigiani} 

% El nombre del director y co-director, se puede usar el cualquier lugar del documento con el comando \supname y \cosupname y \pertesupname y \pertecosupname
\director{Urbano Pasquinel}
\pertenenciaDirector{Globant} 
% FIXME:NO IMPLEMENTADO EL CODIRECTOR ni su pertenencia
\codirector{} % si queda vacio no se deberíá incluir 
\pertenenciaCoDirector{}

% Nombre del cliente, quien va a aprobar los resultados del proyecto, se puede usar con el comando \clientename y \empclientename
\cliente{Globant}
\empresaCliente{Globant}

% Nombre y pertenencia de los jurados, se pueden usar el cualquier lugar del documento con el comando \jurunoname, \jurdosname y \jurtresname y \perteunoname, \pertedosname y \pertetresname.
\juradoUno{Nombre y Apellido (1)}
\pertenenciaJurUno{pertenencia (1)} 
\juradoDos{Nombre y Apellido (2)}
\pertenenciaJurDos{pertenencia (2)}
\juradoTres{Nombre y Apellido (3)}
\pertenenciaJurTres{pertenencia (3)}
 
\fechaINICIO{5 de marzo de 2021}		%Fecha de inicio de la cursada de GdP \fechaInicioName
\fechaFINALPlanificacion{23 de abril de 2021} 	%Fecha de final de cursada de GdP
\fechaFINALTrabajo{---}		%Fecha de defensa pública del trabajo final


\begin{document}

\maketitle
\thispagestyle{empty}
\pagebreak


\thispagestyle{empty}
{\setlength{\parskip}{0pt}
\tableofcontents{}
}
\pagebreak


\section{Registros de cambios}
\label{sec:registro}


\begin{table}[ht]
\label{tab:registro}
\centering
\begin{tabularx}{\linewidth}{@{}|c|X|c|@{}}
\hline
\rowcolor[HTML]{C0C0C0} 
Revisión & {1}{c|}{\cellcolor[HTML]{C0C0C0}Detalles de los cambios realizados} & Fecha      \\ \hline
1.0      & Creación del documento                                          & 15/03/2021 \\ \hline
1.1      & Avances de la planificación & 17/03/2021 \\ \hline
%1.2      & Otro ejemplo \newline
%		   Con texto partido \newline
%		   En varias líneas \newline
%		   A propósito                                                     %& dd/mm/aaaa \\ \hline
\end{tabularx}
\end{table}

\pagebreak



\section{Acta de constitución del proyecto}
\label{sec:acta}

\begin{flushright}
Buenos Aires, \fechaInicioName
\end{flushright}

\vspace{2cm}

Por medio de la presente se acuerda con el Ing. \authorname\hspace{1px} que su Trabajo Final de la \degreename\hspace{1px} se titulará ``\ttitle'', consistirá esencialmente en utilizar computer vision y deep learning para obtener información sobre el comportamiento de personas/clientes dentro de un recinto, y tendrá un presupuesto preliminar estimado de 600 hs de trabajo, con fecha de inicio \fechaInicioName\hspace{1px} y fecha de presentación pública \fechaFinalName.

Se adjunta a esta acta la planificación inicial.

\vfill

% Esta parte se construye sola con la información que hayan cargado en el preámbulo del documento y no debe modificarla
\begin{table}[ht]
\centering
\begin{tabular}{ccc}
\begin{tabular}[c]{@{}c@{}}Ariel Lutenberg \\ Director posgrado FIUBA\end{tabular} & \hspace{2cm} & \begin{tabular}[c]{@{}c@{}}\clientename \\ \empclientename \end{tabular} \vspace{2.5cm} \\ 
\multicolumn{3}{c}{\begin{tabular}[c]{@{}c@{}} \supname \\ Director del Trabajo Final\end{tabular}} \vspace{2.5cm} \\
%\begin{tabular}[c]{@{}c@{}}\jurunoname \\ Jurado del Trabajo Final\end{tabular}     &  & \begin{tabular}[c]{@{}c@{}}\jurdosname\\ Jurado del Trabajo Final\end{tabular}  \vspace{2.5cm}  \\
%\multicolumn{3}{c}{\begin{tabular}[c]{@{}c@{}} \jurtresname\\ Jurado del Trabajo Final\end{tabular}} \vspace{.5cm}                                                                     
\end{tabular}
\end{table}


\newpage

\section{Descripción técnica-conceptual del proyecto a realizar}
\label{sec:descripcion}

El sistema de monitoreo de comportamiento de personas (PBT) tiene como objetivo estudiar los movimientos que realiza una persona al ingresar a un espacio, a fin de obtener métricas sobre los lugares del local que visitó, cuánto tiempo permaneció en el recinto y si transitó por alguna zona de interés predefinida.

Para poder cumplir con el objetivo de estudiar el comportamiento de las personas dentro de un espacio es necesario poder detectar a las distintas personas en el, realizar un seguimiento de cada una y poder identificarlas aún cuando desaparecen del espacio de visión por unos segundos.

Para poder alcanzar el objetivo planteado se debe construir un pipeline de inteligencia artificial como se observa en la Figura \ref{fig:diagBloques}, el cual está comformado por un detector de personas, un tracker (seguidor) y un sistema que permita extraer características (embeddings) que luego el Engine utilizará para poder identificar personas que hayan salido del rango de visión por oclusiones o fuera de imagen.

\vspace{25px}

\begin{figure}[htpb]
\centering 
\includegraphics[width=.7\textwidth]{./Figuras/diagBloques.png}
\caption{Diagrama en bloques del sistema}
\label{fig:diagBloques}
\end{figure}

El objetivo principal del Engine es identificar a cada persona con un “id” único y analizar sus movimientos dentro del reciento, como también debe resolver las siguientes problemáticas:
\begin{itemize}
\item Resolver que una persona salga y entre del rango de visión, el sistema la debe identificar como la misma persona.
\item Resolver las oclusiones que puedan llegar a ocurrir (personas que tapan a otras personas en el rango de visión).
\item Mantener el id correcto entre las personas, evitar que los “id” se intercambien.
\end{itemize}

\newpage

Toda la información recolectada por el Engine será enviada a una aplicación web desarrollada en Python, la cual mostrará un dashboard con toda la información recolectada al momento. El Engine para poder realizar todo el labor mencionado necesita como entrada las personas detectadas en un video/imagen (la bounding box de cada una), un “id” tentativo entregado por el Tracker, y características (embeddings)  entregadas por un extractor. Para ello ya se estuvo trabajando y ensayando los siguientes modelos pre-entrenados:

\begin{itemize}
\item Detector: Se utilizará el modelo pre-entrenado “Yolo” como detector por excelencia para detectar personas en una imagen/video. Se utilizará la versión completa de Yolo, a fin de obtener la mayor precisión posible.
\item Tracker: Se utilizará el modelo pre-entrenado “DeepSort” como tracker por excelencia para asignar un primer “id” tentativo utilizando flujo óptimo, técnicas de seguimiento de patrones (Mean Shift) y un clasificador básico para resolver oclusiones momentáneas.
\item Extractor de características: Se ensayarán diferentes alternativas como extractor de atributos, extractor de vectores de personas, clasificadores de imágenes entrenados con datasets de personas, etc. Sea cual sea el modelo definitivo que se utilice el objetivo es obtener un vector de características de cada persona detectada que permita al Engine identificar a las personas en el recinto.
\end{itemize}

\textbf{NOTA}: Las redes generadores de vectores (extractor) ya se encuentran pre entrenadas con datasets de personas y los modelos/pesos están disponibles en internet. Por una cuestión académica y reforzar este trabajo se realizará el entrenamiento fine-tuning de la red seleccionada como extractor, para luego comparar contra el modelo utilizado disponible en internet. Los datasets más utilizados para este propósito son el Market 1051 y DukeMTMC-reID.


\section{Identificación y análisis de los interesados}
\label{sec:interesados}

\begin{table}[ht]
%\caption{Identificación de los interesados}
%\label{tab:interesados}
\begin{tabularx}{\linewidth}{@{}|l|X|X|l|@{}}
\hline
\rowcolor[HTML]{C0C0C0} 
Rol           & Nombre y Apellido & Organización 	& Puesto 	\\ \hline
Auspiciante   & Empresa EEUU      & (confidencial) 	&        	\\ \hline
Cliente       & \clientename      &\empclientename	&        	\\ \hline
Impulsor      &                   &              	&        	\\ \hline
Responsable   & \authorname       & FIUBA        	& Alumno 	\\ \hline
Colaboradores &                   &              	&        	\\ \hline
Orientador    & \supname	      & \pertesupname 	& Director	Trabajo final \\ \hline
Equipo        & (confidencial) 	  & Globant         & IoT Engineer   	\\ \hline
Opositores    &                   &              	&        	\\ \hline
Usuario final &                   &              	&        	\\ \hline
\end{tabularx}
\end{table}

Nota: Por motivos de confidencialidad no puedo mencionar el nombre del auspiciante ni de los miembros del equipo.

\newpage

\section{1. Propósito del proyecto}
\label{sec:proposito}

El propósito de este proyecto es obtener métricas de los movimientos que realiza una persona al ingresar a un espacio, a fin de obtener los siguientes indicadores:
\begin{itemize}
\item Determinan cuántas personas se encuentran en el espacio de interés.
\item Determinar zonas de interés en el espacio y determinar cuántas y cuales personas transitaron por este. Las zonas de interés se dibujarán con polígonos en la imagen que representará al espacio de interés.
\item Determinar cuánto tiempo las personas estuvieron dentro del espacio y las diferentes zonas de interés.
\item Determinar cuando una persona abandona el espacio.
\end{itemize}

\section{2. Alcance del proyecto}
\label{sec:alcance}
Todas las pruebas se realizarán sobre videos (mp4) públicos, en donde haya un grupo de personas que permita su estudio de comportamiento. Se utilizarán videos que se acerquen lo mayor posible a un video que podría obtenerse de una cámara de seguridad de una tienda o un espacio. En caso de que la empresa consiga videos que puedan ayudar al problema, estos no podrán compartirse a menos que la empresa lo permita y en todo caso sólo se mostrarán los resultados o métricas alcanzadas en el dashboard. El dashboard \textbf{no mostrará el streaming de video} sino una representación del espacio y las métricas sobre el comportamiento de las personas monitoreadas.

Para la elaboración de este proyecto la parte del pipeline correspondiente al Detector + Tracker + Extractor se ejecutará en Colab. Se consumirá un video pregrabado mp4 el cual generará un archivo JSON con todos los datos recolectados del video que serán los inputs para el Engine para que este luego los consuma finalizado este proceso. El Engine correrá en una máquina local o dispositivo dedicado utilizando como entrada el archivo de datos, y la aplicación web dashboard  en una máquina local o servicio en la nube conectada al Engine en real-time (a medida que el Engine consume el archivo y genera información, los resultados se verán reflejados en la App).

\begin{figure}[htpb]
\centering 
\includegraphics[width=.7\textwidth]{./Figuras/diagEjecucion.png}
\caption{Diagrama de ejecución del sistema}
\label{fig:diagEjecucion}
\end{figure}

\newpage

\section{3. Supuestos del proyecto}
\label{sec:supuestos}

Para el desarrollo del presente proyecto se supone que:

\begin{itemize}
\item Todo el proceso que requiera GPU será ejecutado en Colab (entorno virtual gratuito de Google para ejecutar modelos de IA).
\item Todo el proceso ejecutado en Colab no será real-time, la salida del sistema será un archivo que consuma luego el Engine.
\item En esta fase no se invertirá en hardware, se contará con Colab para el procesamiento pesado en GPU.
\item En esta etapa del proyecto no se realizará el deploy del sistema en ningún sistema embebido o servidor, se mantendrá todo en un entorno controlado de desarrollo (Colab + computadora personal).
\item Se utilizá diferentes videos que representen un escenario clásico de un local, tienda, recinto.
\item En esta fase no será requerido realizar corrección de imagen o deformación de los videos adquiridos, se buscará videos sin deformaciones.
\item No se espera realizar el correcto seguimiento del 100\% de las personas, se establecerá un margen de aceptación para ello.
\item No se espera evitar problemas que surjan por movimientos o comportamientos de las personas que salgan de lo normal esperado.
\end{itemize}

\newpage

\section{4. Requerimientos}
\label{sec:requerimientos}

\begin{enumerate}
\item Grupo de requerimientos asociados con el pipeline de IA:
	\begin{enumerate}
	\item Detectar y seguir personas en un video, excluir otros elementos.
	\item Generar características que permitan luego re identificar personas perdidas.
	\item El sistema debe entregar como resultado un archivo con los datos de detección, seguimiento y características en formato JSON.
	\end{enumerate}
\item Grupo de requerimientos asociados con el Engine:
	\begin{enumerate}
	\item Consumir e interpretar los datos provenientes del pipeline de IA.
	\item Resolver problemáticas en el seguimiento de las personas utilizando sus características.
	\item Medir el comportamiento de cada persona con diferentes áreas de interés definidas en la aplicación.
	\item El sistema debe enviar los resultados en un JSON con toda la información de interés a ser consumida por la aplicación web.
	\end{enumerate}
\item Grupo de requerimientos asociados con la aplicación web:
\begin{enumerate}
	\item Mediante una interfaz web poder definir las zonas de interés en el espacio.
	\item Consumir e interpretar los datos provenientes del Engine.
	\item Se considerará que una persona es correctamente monitoreada si al menos se mantuvo su seguimiento el 80\% del tiempo que circuló en el recinto.
	\item Se considerará que el sistema funciona dentro de los parámetros aceptables si entre el 80\% y 100\% de las personas en el video fueron correctamente monitoreadas.
	\item Poder observar en tiempo real los datos que se obtienen de cada persona y las zonas definidas.
	\end{enumerate}
\item Grupo de requerimientos asociados con regulaciones:
\begin{enumerate}
	\item Los videos utilizados no inflijan derechos de privacidad.
	\item El sistema no asociará las personas en seguimiento con un persona física real.
	\item Las personas en seguimiento se harán referencia con un número incremental asociado a la base de datos, sin almacenar ningún tipo de información privada.
	\end{enumerate}
\end{enumerate}

\section{Historias de usuarios (\textit{Product backlog})}
\label{sec:backlog}

\begin{consigna}{red}
Descripción: En esta sección se deben incluir las historias de usuarios y su ponderación (\textit{history points}). Recordar que las historias de usuarios son descripciones cortas y simples de una característica contada desde la perspectiva de la persona que desea la nueva capacidad, generalmente un usuario o cliente del sistema. La ponderación es un número entero que representa el tamaño de la historia comparada con otras historias de similar tipo.
\end{consigna}

\section{5. Entregables principales del proyecto}
\label{sec:entregables}

\begin{itemize}
\item Esquema del funcionamiento de cada pieza del pipeline de IA de Colab (código reservado para Globant).
\item Esquema del funcionamiento del Engine de seguimiento (código reservado para Globant).
\item Aplicación web para interactuar con el sistema y ejecutar los ensayos.
\item Informe final
\end{itemize}

\section{6. Desglose del trabajo en tareas}
\label{sec:wbs}

\begin{enumerate}
\item Desarrollo del pipeline IA. (368hs)
	\begin{enumerate}
	\item Obtener y analizar videos. (48hs)
	\item Ensayo de modelos de detección y trackeo. (40hs)
	\item Ensayo del extractor de características DeepMAR. (40hs)
	\item Ensayo del extractor de características OsNet. (40hs)
	\item Integración del extractor de características en el pipeline de detección. (40hs)
	\item Ensayos de integración profunda del extractor con el tracker. (40hs)
	\item Integrar al pipeline un modelo para la predicción de poses de la persona. (20hs)
	\item Armar el dataset para el entrenamiento del modelo extractor con Tensorflow (TF). (40hs)
	\item Entrenar y ensayar el modelo extractor en TF (TF). (40hs)
	\item Comparativas entre los modelos extractores entrenados con los modelos pre-entrenados obtenidos en la web. (20hs)	
	\end{enumerate}
\item Desarrollo del Engine de seguimiento. (160hs)
	\begin{enumerate}
	\item Desarrollo del sistema de clustering para la re identificación de personas. (40hs)
	\item Definir las zonas de interés a partir de un JSON o archivo. (20hs)
	\item Ubicar las personas en el plano relativo a la imagen, a fin de determinar su ubicación en el recinto (20hs)
	
	\item Realizar lógica de detección dentro de zonas. (20hs)
	\item Desarrollar las APIs para integrar el Engine con dashboard o aplicación web. (20hs)
	\item Mejorar el accuracy del sistema analizando casos de borde (40hs)	
	\end{enumerate}
\item Desarrollo del dasboard/web app. (140hs)
	\begin{enumerate}
	\item Desarrollar la interfaz de usuario para definir las zonas de interés sobre una imagen (40hs)
	\item Desarrollar las APIs de comunicación con el Engine. (20hs)
	\item Desarrollar la interfaz de visualización de las métricas de las personas. (40hs)
	\item Desarrollar la interfaz de visualización de la representación de las personas en la tienda en tiempo real. (40hs)
	\end{enumerate}
\item Presentación del trabajo. (100hs)
	\begin{enumerate}
	\item Redacción del informe de avance. (20hs)
	\item Redacción de las memorias del proyecto. (60hs)
	\item Preparación de la presentación pública. (20hs)
	\end{enumerate}
\end{enumerate}

Cantidad total de horas: (768hs)


\section{7. Diagrama de Activity On Node}
\label{sec:AoN}

\begin{consigna}{red}
Armar el AoN a partir del WBS definido en la etapa anterior. 

%La figura \ref{fig:AoN} fue elaborada con el paquete latex tikz y pueden consultar la siguiente referencia \textit{online}:

%\url{https://www.overleaf.com/learn/latex/LaTeX_Graphics_using_TikZ:_A_Tutorial_for_Beginners_(Part_3)\%E2\%80\%94Creating_Flowcharts}

\end{consigna}

\begin{figure}[htpb]
\centering 
\includegraphics[width=.8\textwidth]{./Figuras/AoN.png}
\caption{Diagrama en \textit{Activity on Node}}
\label{fig:AoN}
\end{figure}

Indicar claramente en qué unidades están expresados los tiempos.
De ser necesario indicar los caminos semicríticos y analizar sus tiempos mediante un cuadro.
Es recomendable usar colores y un cuadro indicativo describiendo qué representa cada color, como se muestra en el siguiente ejemplo:



\section{8. Diagrama de Gantt}
\label{sec:gantt}

\begin{consigna}{red}
Utilizar el software Gantter for Google Drive o alguno similar para dibujar el diagrama de Gantt.

Existen muchos programas y recursos \textit{online} para hacer diagramas de gantt, entre las cuales destacamos:

\begin{itemize}
\item Planner
\item GanttProject
\item Trello + \textit{plugins}. En el siguiente link hay un tutorial oficial: \\ \url{https://blog.trello.com/es/diagrama-de-gantt-de-un-proyecto}
\item Creately, herramienta online colaborativa. \\\url{https://creately.com/diagram/example/ieb3p3ml/LaTeX}
\item Se puede hacer en latex con el paquete \textit{pgfgantt}\\ \url{http://ctan.dcc.uchile.cl/graphics/pgf/contrib/pgfgantt/pgfgantt.pdf}
\end{itemize}

Pegar acá una captura de pantalla del diagrama de Gantt, cuidando que la letra sea suficientemente grande como para ser legible. 
Si el diagrama queda demasiado ancho, se puede pegar primero la ``tabla'' del Gantt y luego pegar la parte del diagrama de barras del diagrama de Gantt.

Configurar el software para que en la parte de la tabla muestre los códigos del EDT (WBS).\\
Configurar el software para que al lado de cada barra muestre el nombre de cada tarea.\\
Revisar que la fecha de finalización coincida con lo indicado en el Acta Constitutiva.

En la figura \ref{fig:gantt}, se muestra un ejemplo de diagrama de gantt realizado con el paquete de \textit{pgfgantt}. En la plantilla pueden ver el código que lo genera y usarlo de base para construir el propio.

\begin{figure}[htbp]
\begin{center}
\begin{ganttchart}{1}{12}
  \gantttitle{2020}{12} \\
  \gantttitlelist{1,...,12}{1} \\
  \ganttgroup{Group 1}{1}{7} \\
  \ganttbar{Task 1}{1}{2} \\
  \ganttlinkedbar{Task 2}{3}{7} \ganttnewline
  \ganttmilestone{Milestone o hito}{7} \ganttnewline
  \ganttbar{Final Task}{8}{12}
  \ganttlink{elem2}{elem3}
  \ganttlink{elem3}{elem4}
\end{ganttchart}
\end{center}
\caption{Diagrama de gantt de ejemplo}
\label{fig:gantt}
\end{figure}

\end{consigna}

\section{9. Matriz de uso de recursos de materiales}
\label{sec:recursos}


\begin{table}
\label{tab:recursos}
\centering
\begin{tabularx}{\linewidth}{@{}|c|X|X|X|X|c|@{}}
\hline
\cellcolor[HTML]{C0C0C0} & \cellcolor[HTML]{C0C0C0} & \multicolumn{4}{c|}{\cellcolor[HTML]{C0C0C0}Recursos requeridos (horas)} \\ \cline{3-6} 
\multirow{-2}{*}{\cellcolor[HTML]{C0C0C0}\begin{tabular}[c]{@{}c@{}}Código\\ WBS\end{tabular}} & \multirow{-2}{*}{\cellcolor[HTML]{C0C0C0}\begin{tabular}[c]{@{}c@{}}Nombre \\ tarea\end{tabular}} & Material 1 & Material 2 & Material 3 & Material 4 \\ \hline
 &  &  &  &  &  \\ \hline
 &  &  &  &  &  \\ \hline
 &  &  &  &  &  \\ \hline
 &  &  &  &  &  \\ \hline
 &  &  &  &  &  \\ \hline
 &  &  &  &  &  \\ \hline
 &  &  &  &  &  \\ \hline
 &  &  &  &  &  \\ \hline 
 &  &  &  &  &  \\ \hline
 &  &  &  &  &  \\ \hline
 &  &  &  &  &  \\ \hline
 &  &  &  &  &  \\ \hline
 &  &  &  &  &  \\ \hline
 &  &  &  &  &  \\ \hline
 &  &  &  &  &  \\ \hline
 &  &  &  &  &  \\ \hline
 &  &  &  &  &  \\ \hline
 &  &  &  &  &  \\ \hline
 &  &  &  &  &  \\ \hline
 &  &  &  &  &  \\ \hline
 &  &  &  &  &  \\ \hline
 &  &  &  &  &  \\ \hline
 &  &  &  &  &  \\ \hline
 &  &  &  &  &  \\ \hline 
 &  &  &  &  &  \\ \hline
 &  &  &  &  &  \\ \hline
 &  &  &  &  &  \\ \hline
 &  &  &  &  &  \\ \hline

\end{tabularx}%
\end{table}


\section{10. Presupuesto detallado del proyecto}
\label{sec:presupuesto}

\begin{consigna}{red}
Si el proyecto es complejo entonces separarlo en partes:
\begin{itemize}
\item Un total global, indicando el subtotal acumulado por cada una de las áreas.
\item El desglose detallado del subtotal de cada una de las áreas.
\end{itemize}

IMPORTANTE: No olvidarse de considerar los COSTOS INDIRECTOS.

\end{consigna}

\begin{table}[htpb]
\centering
\begin{tabularx}{\linewidth}{@{}|X|c|r|r|@{}}
\hline
\rowcolor[HTML]{C0C0C0} 
\multicolumn{4}{|c|}{\cellcolor[HTML]{C0C0C0}COSTOS DIRECTOS} \\ \hline
\rowcolor[HTML]{C0C0C0} 
Descripción &
  \multicolumn{1}{c|}{\cellcolor[HTML]{C0C0C0}Cantidad} &
  \multicolumn{1}{c|}{\cellcolor[HTML]{C0C0C0}Valor unitario} &
  \multicolumn{1}{c|}{\cellcolor[HTML]{C0C0C0}Valor total} \\ \hline
 &
  \multicolumn{1}{c|}{} &
  \multicolumn{1}{c|}{} &
  \multicolumn{1}{c|}{} \\ \hline
 &
  \multicolumn{1}{c|}{} &
  \multicolumn{1}{c|}{} &
  \multicolumn{1}{c|}{} \\ \hline
\multicolumn{1}{|l|}{} &
   &
   &
   \\ \hline
\multicolumn{1}{|l|}{} &
   &
   &
   \\ \hline
\multicolumn{3}{|c|}{SUBTOTAL} &
  \multicolumn{1}{c|}{} \\ \hline
\rowcolor[HTML]{C0C0C0} 
\multicolumn{4}{|c|}{\cellcolor[HTML]{C0C0C0}COSTOS INDIRECTOS} \\ \hline
\rowcolor[HTML]{C0C0C0} 
Descripción &
  \multicolumn{1}{c|}{\cellcolor[HTML]{C0C0C0}Cantidad} &
  \multicolumn{1}{c|}{\cellcolor[HTML]{C0C0C0}Valor unitario} &
  \multicolumn{1}{c|}{\cellcolor[HTML]{C0C0C0}Valor total} \\ \hline
\multicolumn{1}{|l|}{} &
   &
   &
   \\ \hline
\multicolumn{1}{|l|}{} &
   &
   &
   \\ \hline
\multicolumn{1}{|l|}{} &
   &
   &
   \\ \hline
\multicolumn{3}{|c|}{SUBTOTAL} &
  \multicolumn{1}{c|}{} \\ \hline
\rowcolor[HTML]{C0C0C0}
\multicolumn{3}{|c|}{TOTAL} &
   \\ \hline
\end{tabularx}%
\end{table}


\section{11. Matriz de asignación de responsabilidades}
\label{sec:responsabilidades}
\begin{consigna}{red}
Establecer la matriz de asignación de responsabilidades y el manejo de la autoridad completando la siguiente tabla:

\begin{table}[htpb]
\centering
\resizebox{\textwidth}{!}{%
\begin{tabular}{|c|c|c|c|c|c|}
\hline
\rowcolor[HTML]{C0C0C0} 
\cellcolor[HTML]{C0C0C0} &
  \cellcolor[HTML]{C0C0C0} &
  \multicolumn{4}{c|}{\cellcolor[HTML]{C0C0C0}Listar todos los nombres y roles del proyecto} \\ \cline{3-6} 
\rowcolor[HTML]{C0C0C0} 
\cellcolor[HTML]{C0C0C0} &
  \cellcolor[HTML]{C0C0C0} &
  Responsable &
  Orientador &
  Equipo &
  Cliente \\ \cline{3-6} 
\rowcolor[HTML]{C0C0C0} 
\multirow{-3}{*}{\cellcolor[HTML]{C0C0C0}\begin{tabular}[c]{@{}c@{}}Código\\ WBS\end{tabular}} &
  \multirow{-3}{*}{\cellcolor[HTML]{C0C0C0}Nombre de la tarea} &
  \authorname &
  \supname &
  Nombre de alguien &
  \clientename \\ \hline
 &  &  &  &  &  \\ \hline
 &  &  &  &  &  \\ \hline
 &  &  &  &  &  \\ \hline
\end{tabular}%
}
\end{table}

{\footnotesize
Referencias:
\begin{itemize}
	\item P = Responsabilidad Primaria
	\item S = Responsabilidad Secundaria
	\item A = Aprobación
	\item I = Informado
	\item C = Consultado
\end{itemize}
} %footnotesize

Una de las columnas debe ser para el Director, ya que se supone que participará en el proyecto.
A su vez se debe cuidar que no queden muchas tareas seguidas sin ``A'' o ``I''.

Importante: es redundante poner ``I/A'' o ``I/C'', porque para aprobarlo o responder consultas primero la persona debe ser informada.

\end{consigna}

\section{12. Gestión de riesgos}
\label{sec:riesgos}

\begin{consigna}{red}
a) Identificación de los riesgos (al menos cinco) y estimación de sus consecuencias:
 
Riesgo 1: detallar el riesgo (riesgo es algo que si ocurre altera los planes previstos)
\begin{itemize}
\item Severidad (S): mientras más severo, más alto es el número (usar números del 1 al 10).\\
Justificar el motivo por el cual se asigna determinado número de severidad (S).
\item Probabilidad de ocurrencia (O): mientras más probable, más alto es el número (usar del 1 al 10).\\
Justificar el motivo por el cual se asigna determinado número de (O). 
\end{itemize}   

Riesgo 2:
\begin{itemize}
\item Severidad (S): 
\item Ocurrencia (O):
\end{itemize}

Riesgo 3:
\begin{itemize}
\item Severidad (S): 
\item Ocurrencia (O):
\end{itemize}


b) Tabla de gestión de riesgos:      (El RPN se calcula como RPN=SxO)

\begin{table}[htpb]
\centering
\begin{tabularx}{\linewidth}{@{}|X|c|c|c|c|c|c|@{}}
\hline
\rowcolor[HTML]{C0C0C0} 
Riesgo & S & O & RPN & S* & O* & RPN* \\ \hline
       &   &   &     &    &    &      \\ \hline
       &   &   &     &    &    &      \\ \hline
       &   &   &     &    &    &      \\ \hline
       &   &   &     &    &    &      \\ \hline
       &   &   &     &    &    &      \\ \hline
\end{tabularx}%
\end{table}

Criterio adoptado: 
Se tomarán medidas de mitigación en los riesgos cuyos números de RPN sean mayores a...

Nota: los valores marcados con (*) en la tabla corresponden luego de haber aplicado la mitigación.

c) Plan de mitigación de los riesgos que originalmente excedían el RPN máximo establecido:
 
Riesgo 1: plan de mitigación (si por el RPN fuera necesario elaborar un plan de mitigación).
  Nueva asignación de S y O, con su respectiva justificación:
  - Severidad (S): mientras más severo, más alto es el número (usar números del 1 al 10).
          Justificar el motivo por el cual se asigna determinado número de severidad (S).
  - Probabilidad de ocurrencia (O): mientras más probable, más alto es el número (usar del 1 al 10).
          Justificar el motivo por el cual se asigna determinado número de (O).

Riesgo 2: plan de mitigación (si por el RPN fuera necesario elaborar un plan de mitigación).
 
Riesgo 3: plan de mitigación (si por el RPN fuera necesario elaborar un plan de mitigación).

\end{consigna}


\section{13. Gestión de la calidad}
\label{sec:calidad}

\begin{consigna}{red}
Para cada uno de los requerimientos del proyecto indique:
\begin{itemize} 
\item Req \#1: copiar acá el requerimiento.

Verificación y validación:

\begin{itemize}
\item Verificación para confirmar si se cumplió con lo requerido antes de mostrar el sistema al cliente. Detallar 
\item Validación con el cliente para confirmar que está de acuerdo en que se cumplió con lo requerido. Detallar  
\end{itemize}

\end{itemize}

Tener en cuenta que en este contexto se pueden mencionar simulaciones, cálculos, revisión de hojas de datos, consulta con expertos, mediciones, etc.

\end{consigna}

\section{14. Comunicación del proyecto}
\label{sec:comunicaciones}

El plan de comunicación del proyecto es el siguiente:

\begin{table}[htpb]
\centering
\begin{tabularx}{\linewidth}{@{}|X|C{2.4cm}|C{3cm}|C{1.8cm}|C{2cm}|C{2.1cm}|@{}}
\hline
\rowcolor[HTML]{C0C0C0} 
\multicolumn{6}{|c|}{\cellcolor[HTML]{C0C0C0}PLAN DE COMUNICACIÓN DEL PROYECTO}           \\ \hline
\rowcolor[HTML]{C0C0C0} 
¿Qué comunicar? & Audiencia & Propósito & Frecuencia & Método de comunicac. & Responsable \\ \hline
                &           &           &            &                      &             \\ \hline
                &           &           &            &                      &             \\ \hline
                &           &           &            &                      &             \\ \hline
                &           &           &            &                      &             \\ \hline
                &           &           &            &                      &             \\ \hline
\end{tabularx}
\end{table}

\section{15. Gestión de compras}
\label{sec:compras}

\begin{consigna}{red}
En caso de tener que comprar elementos o contratar servicios:
a) Explique con qué criterios elegiría a un proveedor.
b) Redacte el Statement of Work correspondiente.
\end{consigna}

\section{16. Seguimiento y control}
\label{sec:seguimiento}

\begin{consigna}{red}
Para cada tarea del proyecto establecer la frecuencia y los indicadores con los se seguirá su avance y quién será el responsable de hacer dicho seguimiento y a quién debe comunicarse la situación (en concordancia con el Plan de Comunicación del proyecto).

El indicador de avance tiene que ser algo medible, mejor incluso si se puede medir en \% de avance. Por ejemplo,se pueden indicar en esta columna cosas como ``cantidad de conexiones ruteadeas'' o ``cantidad de funciones implementadas'', pero no algo genérico y ambiguo como ``\%'', porque el lector no sabe porcentaje de qué cosa.

\end{consigna}

\begin{longtable}{|m{1cm}|m{3.5cm}|m{2.2cm}|m{2cm}|m{3cm}|m{1.5cm}|}
\hline
\rowcolor[HTML]{C0C0C0} 
\multicolumn{6}{|c|}{\cellcolor[HTML]{C0C0C0}SEGUIMIENTO DE AVANCE}                                                                       \\ \hline
\rowcolor[HTML]{C0C0C0} 
Tarea del WBS 			& Indicador de avance & Frecuencia de reporte & Resp. de seguimiento & Persona a ser informada & Método de comunic. \\ \hline
\endfirsthead

\hline
\rowcolor[HTML]{C0C0C0} 
\multicolumn{6}{c}{\cellcolor[HTML]{C0C0C0}SEGUIMIENTO DE AVANCE}                                                                       \\ \hline
\rowcolor[HTML]{C0C0C0} 
Tarea del WBS 			& Indicador de avance & Frecuencia de reporte & Resp. de seguimiento & Persona a ser informada & Método de comunic. \\ \hline
\endhead

\multicolumn{6}{c}{Continúa}
\endfoot

\endlastfoot

1.1	& Fecha de inicio  & Única vez al comienzo & \authorname & \clientename, \supname & email \\ \hline
2.1	& Avance de las subtareas  & Mensual mientras dure la tarea & \authorname & \clientename, \supname & email \\ \hline

\end{longtable}

\begin{table}[!htpb]
\centering
%\begin{tabularx}{\linewidth}{@{}|X|X|X|X|X|X|@{}}
\begin{tabularx}{\linewidth}{@{}|X|C{2.5cm}|C{3cm}|C{2cm}|C{2cm}|C{2.5cm}|@{}}
\hline
\rowcolor[HTML]{C0C0C0} 
\multicolumn{6}{|c|}{\cellcolor[HTML]{C0C0C0}SEGUIMIENTO DE AVANCE}                                                                       \\ \hline
\rowcolor[HTML]{C0C0C0} 
Tarea del WBS & Indicador de avance & Frecuencia de reporte & Resp. de seguimiento & Persona a ser informada & Método de comunic. \\ \hline
 &  &  &  &  &  \\ \hline
 &  &  &  &  &  \\ \hline
 &  &  &  &  &  \\ \hline
 &  &  &  &  &  \\ \hline
 &  &  &  &  &  \\ \hline
\end{tabularx}%
%}
\end{table}

\section{17. Procesos de cierre}    
\label{sec:cierre}

\begin{consigna}{red}
Establecer las pautas de trabajo para realizar una reunión final de evaluación del proyecto, tal que contemple las siguientes actividades:

\begin{itemize}
\item Pautas de trabajo que se seguirán para analizar si se respetó el Plan de Proyecto original:
 - Indicar quién se ocupará de hacer esto y cuál será el procedimiento a aplicar. 
\item Identificación de las técnicas y procedimientos útiles e inútiles que se utilizaron, y los problemas que surgieron y cómo se solucionaron:
 - Indicar quién se ocupará de hacer esto y cuál será el procedimiento para dejar registro.
\item Indicar quién organizará el acto de agradecimiento a todos los interesados, y en especial al equipo de trabajo y colaboradores:
  - Indicar esto y quién financiará los gastos correspondientes.
\end{itemize}

\end{consigna}


\end{document}
